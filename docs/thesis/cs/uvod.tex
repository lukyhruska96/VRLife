\chapter{Úvod}

Sociální sítě jsou pro většinu lidí neodmyslitelnou součástí každého dne.
Dávají nám totiž možnost komunikovat s přáteli, sdílet s nimi své zážitky
a to bez ohledu na to, jak daleko od vás právě jsou.

Aktuálně však narážíme na myšlenku, že většina sociálních sítí vypadají téměř stejně,
jako před deseti lety a přitom máme dostatek technologií, abychom ještě více přiblížili svět
sociálních sítí k běžné komunikaci.

Této myšlenky se chytil již v roce 2014 Graham Gaylor a Jesse Joudrey, kdy spolu vytvořili
hru s názvem VrChat. Jedná se o free-to-play masově multiplayerovou hru, kde můžete interagoat
s ostatními pomocí své 3D postavy. Původně se jednalo o hru pouze pro majitele VR headsetu, nyní
však VR headset není potřeba a tím se i tato hra otevřela širší veřejnosti. Výhodou této hry
je její masivní API, díky kterému můžete vytvářet nové světy/místnosti, postavy i hry.
A právě tato modularita s 3D světem ukázala nové možnosti moderní sociální sítě.

\section{Témata k pokrytí}
TATO SEKCE BUDE PŘED ODEVZDÁNÍM SMAZÁNA
JEDNÁ SE POUZE O SHRNUTÍ TÉMAT, KTERÁ BYCH CHTĚL POKRÝT V TÉTO PRÁCI
\begin{itemize}
    \item Způsob nasazení serveru (regionální provideři)
    \item Úpravy k podpoře VR
\end{itemize}
\section{Modulární aplikace}

\section{Sociální sítě}

\section{Shrnutí cílů práce}